
% ===================================
%  document formatting refinement
% ===================================

% No indents for whole text
\setlength{\parindent}{0pt}

% improve the line breaking 
\setlength{\emergencystretch}{1em}

% pic width = \textwidth - \chapterpicwidth
\def\chapterpicwidth{0.43}

% line breaks for URLs
\renewcommand{\UrlBreaks}{\do\/}


%
% custom hyphenation
%
\hyphenation{in-stru-men-ta-tion analy-sis con-cur-ren-cy}


% ===================================
%  settings for the listings package
% ===================================

\definecolor{lightGray}{RGB}{242,242,242}
\lstset{
  language=C++,
  basicstyle=\ttfamily,
  columns=fullflexible,
  keepspaces=true, % To have spaces monsized
  tabsize=2,
  frame=none, % = single
  breaklines=true,
  numbers=left,
  numberstyle=\tiny,
  escapeinside={//@}{@},
  showstringspaces=false,
  resetmargins=true,
  xleftmargin=18pt,
  xrightmargin=0pt,
  %resetmargins=true,
  %framextopmargin=0pt
  backgroundcolor=\color{lightGray},
  captionpos=b,
  %abovecaptionskip=0pt,
  %belowcaptionskip=0pt,
  %aboveskip=3pt,
  %belowskip=0pt,
  %framesep=1pt,
  numberbychapter=false,
}

% coloring for listings
\lstdefinestyle{colored}{
  moredelim=[is][\color{blue}]{@}{@},
  moredelim=[is][\color{red}]{|}{|},
}

% 
% Counter and caption for listings
%

%\newcounter{lstlisting}  % already defined by pkg listing at \begin{document}
\counterwithout{lstlisting}{chapter}

\newcommand{\inputlisting}[1]{  
  \begin{minipage}{\columnwidth}
  \lstinputlisting{#1}
  \end{minipage}
}

\newcommand{\lstgname}{Listing}
\newcommand{\listingcaption}[1]% 
{%
  {%
    \refstepcounter{lstlisting}
    \noindent\footnotesize{\lstgname~\thelstlisting{:}~#1\hfill}
  }%
}%



% ===========================
%  cover generation commands
% ===========================

\def\backcover{
\genbackpage{
  \textbf{Siemens AG}\newline
  Corporate Technology\newline
  Multicore Expert Center\newline
  \newline
  Otto-Hahn-Ring 6\newline
  81739 Muenchen\newline
  Germany\newline
  \newline
  {http://multicore.ct.siemens.de}
  }
}
% ======================
%  convenience commands
% ======================

\def\chaptername{Chapter}
\def\algname{Algorithm}
\def\sectionname{Section}

%
% Programming languages and frameworks
%

\def\qt{Qt}
\def\cE{C11}
\def\cpp{C\raisebox{0.17ex}{\small\textbf{++}}}
\def\cppFootnote{C\raisebox{0.08ex}{\small{++}}}
\def\cppE{{\cpp}11}
\newcommand{\csharp}{%
  {\settoheight{\dimen0}{C}C\kern-.05em\hspace{0.5pt}\resizebox{!}{\dimen0}{\raisebox{\depth}{\#}}}}
\def\dotnet{.NET}

%
% Special characters
%
\def\myCheck{\ding{51}} % check
\def\myCross{\ding{55}} % cross


%
% Various stuff
%
\def\time#1{#1}

\def\markup#1{{\color{NavyBlue}#1}}

\newcommand{\toolcard}[9]{
  \begin{table}
    \caption{Quick card for {#2}}
    {\small
      \begin{tabular}{>{\bfseries}p{0.295\columnwidth}p{0.605\columnwidth}}
        \thickhline 
        {Tool} & #2 \\
        {Detectable bugs} & #3 \\
        {License} & #4 \\
        {Platforms} & #5 \\
        {Operating systems} & #6 \\
        {\mbox{Languages}} & #7 \\
        {Threading libraries} & {#8} \\
        {Additional facts} & {#9} \\
        \thickhline
      \end{tabular}
    }
    \label{qc:#1}
  \end{table}
}

% table cell that allows line breaks (e.g., for toolcard)
%   first (optional) parameter defines alignment (l,c,r)
%   second parameter is the content
\def\lbcell[#1]#2{\rule{0pt}{4ex}\shortstack[#1]{#2}}


% =======================
%  other (various) stuff
% =======================

%
% very custom stuff
%
\def\hb#1{\hbImpl(#1)}
\def\hbRel{\rightarrow}
\def\hbImpl(#1,#2){$#1 \hbRel #2$}
